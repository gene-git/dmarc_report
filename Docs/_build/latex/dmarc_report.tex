%% Generated by Sphinx.
\def\sphinxdocclass{report}
\documentclass[letterpaper,10pt,english]{sphinxmanual}
\ifdefined\pdfpxdimen
   \let\sphinxpxdimen\pdfpxdimen\else\newdimen\sphinxpxdimen
\fi \sphinxpxdimen=.75bp\relax
\ifdefined\pdfimageresolution
    \pdfimageresolution= \numexpr \dimexpr1in\relax/\sphinxpxdimen\relax
\fi
%% let collapsible pdf bookmarks panel have high depth per default
\PassOptionsToPackage{bookmarksdepth=5}{hyperref}

\PassOptionsToPackage{booktabs}{sphinx}
\PassOptionsToPackage{colorrows}{sphinx}

\PassOptionsToPackage{warn}{textcomp}
\usepackage[utf8]{inputenc}
\ifdefined\DeclareUnicodeCharacter
% support both utf8 and utf8x syntaxes
  \ifdefined\DeclareUnicodeCharacterAsOptional
    \def\sphinxDUC#1{\DeclareUnicodeCharacter{"#1}}
  \else
    \let\sphinxDUC\DeclareUnicodeCharacter
  \fi
  \sphinxDUC{00A0}{\nobreakspace}
  \sphinxDUC{2500}{\sphinxunichar{2500}}
  \sphinxDUC{2502}{\sphinxunichar{2502}}
  \sphinxDUC{2514}{\sphinxunichar{2514}}
  \sphinxDUC{251C}{\sphinxunichar{251C}}
  \sphinxDUC{2572}{\textbackslash}
\fi
\usepackage{cmap}
\usepackage[T1]{fontenc}
\usepackage{amsmath,amssymb,amstext}
\usepackage{babel}



\usepackage{tgtermes}
\usepackage{tgheros}
\renewcommand{\ttdefault}{txtt}



\usepackage[Bjarne]{fncychap}
\usepackage{sphinx}

\fvset{fontsize=auto}
\usepackage{geometry}


% Include hyperref last.
\usepackage{hyperref}
% Fix anchor placement for figures with captions.
\usepackage{hypcap}% it must be loaded after hyperref.
% Set up styles of URL: it should be placed after hyperref.
\urlstyle{same}

\addto\captionsenglish{\renewcommand{\contentsname}{Contents:}}

\usepackage{sphinxmessages}
\setcounter{tocdepth}{1}



\title{dmarc\_report}
\date{Feb 23, 2025}
\release{4.13.1}
\author{Gene C}
\newcommand{\sphinxlogo}{\vbox{}}
\renewcommand{\releasename}{Release}
\makeindex
\begin{document}

\ifdefined\shorthandoff
  \ifnum\catcode`\=\string=\active\shorthandoff{=}\fi
  \ifnum\catcode`\"=\active\shorthandoff{"}\fi
\fi

\pagestyle{empty}
\sphinxmaketitle
\pagestyle{plain}
\sphinxtableofcontents
\pagestyle{normal}
\phantomsection\label{\detokenize{index::doc}}


\sphinxstepscope


\chapter{dmarc\_report}
\label{\detokenize{README:dmarc-report}}\label{\detokenize{README::doc}}

\section{Overview}
\label{\detokenize{README:overview}}
\sphinxAtStartPar
Generate a human readable DMARC report from 1 or more standard DMARC and TLS\sphinxhyphen{}RPT xml email reports .

\sphinxAtStartPar
Note:
\begin{quote}

\sphinxAtStartPar
All git tags are signed by \textless{}\sphinxhref{mailto:arch@sapience.com}{arch@sapience.com}\textgreater{}.
Public key is available via WKD or download from website:
\sphinxurl{https://www.sapience.com/tech}
After key is on keyring use the PKGBUILD source line ending with \sphinxstyleemphasis{?signed}
or manually verify using {\color{red}\bfseries{}*}git tag \sphinxhyphen{}v \textless{}tag\sphinxhyphen{}name\textgreater{}
\end{quote}


\section{New / Interesting}
\label{\detokenize{README:new-interesting}}
\sphinxAtStartPar
\sphinxstylestrong{Interesting}
\begin{itemize}
\item {} 
\sphinxAtStartPar
Switch to \sphinxstyleemphasis{py\sphinxhyphen{}cidr} package for handling IPs instead of own versions.
\begin{description}
\sphinxlineitem{Available}\begin{itemize}
\item {} 
\sphinxAtStartPar
github \textless{}\sphinxurl{https://github.com/gene-git/py-cidr}\textgreater{}

\item {} 
\sphinxAtStartPar
AUR \textless{}\sphinxurl{https://aur.archlinux.org/packages/py-cidr}\textgreater{}

\end{itemize}

\end{description}

\item {} 
\sphinxAtStartPar
Now use python 3’s ipaddress module instead of netaddr.
Its faster and we no longer require 3rd party library

\item {} 
\sphinxAtStartPar
Require python version 3.11 or later

\item {} 
\sphinxAtStartPar
Switch to lxml for better handling of xml namespaces found in some reports

\item {} 
\sphinxAtStartPar
Add support for handling mbox file with multiple emails containing reports.
While some clients save multiple emails in separate \sphinxstyleemphasis{.eml} files, others, like
evolution, save them all in a single \sphinxstyleemphasis{.mbox} file. Add support for this.

\item {} 
\sphinxAtStartPar
tls\sphinxhyphen{}rpt

\sphinxAtStartPar
New tool to generate report for TLS reports for MTA\sphinxhyphen{}STS or DANE. See README\sphinxhyphen{}tls.md
This report has been updated \sphinxhyphen{} see Changelog for details.

\item {} 
\sphinxAtStartPar
\sphinxstyleemphasis{\sphinxhyphen{}ifd, \textendash{}inp\_file\_disp}

\sphinxAtStartPar
Input file disposition options one of : none,save,delete
If set to save then all input files (xml, compressed xml and any kept eml files) are moved
to directory specified by \sphinxstyleemphasis{inp\_files\_save\_dir}.

\item {} 
\sphinxAtStartPar
\sphinxstyleemphasis{\sphinxhyphen{}ifsd, \textendash{}inp\_files\_save\_dir}

\sphinxAtStartPar
When \sphinxstyleemphasis{inp\_file\_disp} is set, then input files are moved to this directory after report
is generated.  Files are saved by year\sphinxhyphen{}month under the save directory

\end{itemize}


\chapter{Getting Started}
\label{\detokenize{README:getting-started}}

\section{Installation}
\label{\detokenize{README:installation}}\begin{description}
\sphinxlineitem{Available on}\begin{itemize}
\item {} 
\sphinxAtStartPar
\sphinxhref{https://github.com/gene-git/dmarc\_report}{Github}

\item {} 
\sphinxAtStartPar
\sphinxhref{https://aur.archlinux.org/packages/dmarc\_report}{Archlinux AUR}

\end{itemize}

\end{description}

\sphinxAtStartPar
On Arch you can build using the PKGBUILD provided in packaging directory or from the AUR package.
To build manually, clone the repo and

\begin{sphinxVerbatim}[commandchars=\\\{\}]
rm\PYG{+w}{ }\PYGZhy{}f\PYG{+w}{ }dist/*
python\PYG{+w}{ }\PYGZhy{}m\PYG{+w}{ }build\PYG{+w}{ }\PYGZhy{}\PYGZhy{}wheel\PYG{+w}{ }\PYGZhy{}\PYGZhy{}no\PYGZhy{}isolation
\PYG{n+nv}{root\PYGZus{}dest}\PYG{o}{=}\PYG{l+s+s2}{\PYGZdq{}/\PYGZdq{}}
./scripts/do\PYGZhy{}install\PYG{+w}{ }\PYG{n+nv}{\PYGZdl{}root\PYGZus{}dest}
\end{sphinxVerbatim}

\sphinxAtStartPar
When running as non\sphinxhyphen{}root then set root\_dest a user writable directory


\section{Applications}
\label{\detokenize{README:applications}}
\sphinxAtStartPar
Save all DMARC or TLS\sphinxhyphen{}RPT reports into a directory. These are typically compressed xml files
sent as email attachments.


\subsection{config files}
\label{\detokenize{README:config-files}}
\sphinxAtStartPar
There are 2 config files, \sphinxstyleemphasis{config} for dmarc\sphinxhyphen{}rpt and \sphinxstyleemphasis{tls\sphinxhyphen{}config} for tls\sphinxhyphen{}rpt.

\sphinxAtStartPar
Config files are in a directory which is searched in order:

\begin{sphinxVerbatim}[commandchars=\\\{\}]
\PYG{o}{/}\PYG{n}{etc}\PYG{o}{/}\PYG{n}{dmarc\PYGZus{}report}\PYG{o}{/}\PYG{p}{[}\PYG{n}{tls}\PYG{o}{\PYGZhy{}}\PYG{p}{]}\PYG{n}{config}
\PYG{o}{\PYGZti{}}\PYG{o}{/}\PYG{o}{.}\PYG{n}{config}\PYG{o}{/}\PYG{n}{dmarc\PYGZus{}report}\PYG{o}{/}\PYG{p}{[}\PYG{n}{tls}\PYG{o}{\PYGZhy{}}\PYG{p}{]}\PYG{n}{config}
\end{sphinxVerbatim}

\sphinxAtStartPar
First config found is used.

\sphinxAtStartPar
Config files are standard TOML format.  Available config settings are set using:

\begin{sphinxVerbatim}[commandchars=\\\{\}]
\PYG{n}{command\PYGZus{}line\PYGZus{}long\PYGZus{}opt\PYGZus{}name} \PYG{o}{=} \PYG{n}{xxx}
\end{sphinxVerbatim}

\sphinxAtStartPar
e.g. to set data report dir use:

\begin{sphinxVerbatim}[commandchars=\\\{\}]
\PYG{n+nb}{dir} \PYG{o}{=} \PYG{l+s+s2}{\PYGZdq{}}\PYG{l+s+s2}{/foo/goo/dmarc\PYGZus{}reports}\PYG{l+s+s2}{\PYGZdq{}}
\end{sphinxVerbatim}

\sphinxAtStartPar
Command line options override corresponding config setting.

\sphinxAtStartPar
Example of dmarc \sphinxstyleemphasis{config}:

\begin{sphinxVerbatim}[commandchars=\\\{\}]
\PYG{c+c1}{\PYGZsh{} comment}
\PYG{n+nb}{dir} \PYG{o}{=} \PYG{l+s+s2}{\PYGZdq{}}\PYG{l+s+s2}{\PYGZti{}/dmarc/xml}\PYG{l+s+s2}{\PYGZdq{}}
\PYG{n}{inp\PYGZus{}files\PYGZus{}disp} \PYG{o}{=} \PYG{l+s+s2}{\PYGZdq{}}\PYG{l+s+s2}{save}\PYG{l+s+s2}{\PYGZdq{}}
\PYG{n}{inp\PYGZus{}files\PYGZus{}save\PYGZus{}dir} \PYG{o}{=} \PYG{l+s+s2}{\PYGZdq{}}\PYG{l+s+s2}{../saved}\PYG{l+s+s2}{\PYGZdq{}}
\PYG{n}{dom\PYGZus{}ips} \PYG{o}{=} \PYG{p}{[}\PYG{l+s+s1}{\PYGZsq{}}\PYG{l+s+s1}{1.1.1.1}\PYG{l+s+s1}{\PYGZsq{}}\PYG{p}{,} \PYG{l+s+s1}{\PYGZsq{}}\PYG{l+s+s1}{1.2.2.0/24}\PYG{l+s+s1}{\PYGZsq{}}\PYG{p}{]}
\end{sphinxVerbatim}

\sphinxAtStartPar
This config says to read all the saved email reports from \sphinxstyleemphasis{\textasciitilde{}/dmarc/xml}
and to keep those files after processing report by moving them to \sphinxstyleemphasis{\textasciitilde{}/dmarc/saved}.
It also says that ips listed in dom\_ips are your own domains.

\sphinxAtStartPar
Example of tls\sphinxhyphen{}rpt \sphinxstyleemphasis{tls\sphinxhyphen{}config}:

\begin{sphinxVerbatim}[commandchars=\\\{\}]
\PYG{c+c1}{\PYGZsh{} comment}
\PYG{n+nb}{dir} \PYG{o}{=} \PYG{l+s+s2}{\PYGZdq{}}\PYG{l+s+s2}{\PYGZti{}/tls\PYGZhy{}rpt/xml}\PYG{l+s+s2}{\PYGZdq{}}
\PYG{n}{inp\PYGZus{}files\PYGZus{}disp} \PYG{o}{=} \PYG{l+s+s2}{\PYGZdq{}}\PYG{l+s+s2}{save}\PYG{l+s+s2}{\PYGZdq{}}
\PYG{n}{inp\PYGZus{}files\PYGZus{}save\PYGZus{}dir} \PYG{o}{=} \PYG{l+s+s2}{\PYGZdq{}}\PYG{l+s+s2}{../saved}\PYG{l+s+s2}{\PYGZdq{}}
\end{sphinxVerbatim}

\sphinxAtStartPar
See \sphinxstyleemphasis{Options} section for more detail.


\subsection{dmarc\sphinxhyphen{}rpt Usage}
\label{\detokenize{README:dmarc-rpt-usage}}
\sphinxAtStartPar
Change to the directory containing the one or more dmarc report files and simply run
\begin{quote}

\begin{sphinxVerbatim}[commandchars=\\\{\}]
dmarc\PYGZhy{}rpt
\end{sphinxVerbatim}
\end{quote}

\sphinxAtStartPar
When using the \sphinxstyleemphasis{\textendash{}dir} option (or config setting \sphinxstyleemphasis{dir}) it is not necessary
to change directories before running the report.

\sphinxAtStartPar
Any email files, those ending with \sphinxstyleemphasis{.eml} will be processed first. These are assumed to
contain the report as a mime attachment. The attachment is extracted from any such email
files. Some mail clients save multiple emails as a single mbox file. Each email in the mbox
file will be similarly processed and have the attached report extracted.

\sphinxAtStartPar
Then all remaining files are read and processed. The tool processes all xml
and gzip/zip compressed xml dmarc report files and generates a human readable report.

\sphinxAtStartPar
We follow Postel’s law and try to be liberal in what we accept as input. To that end
we accept the dmarc XML report file, a gzip/zip compressed version of same or a saved email
file text file with the report itself being a mime attachment.

\sphinxAtStartPar
Any file with extension \sphinxstyleemphasis{.eml} is treated as an email file.

\sphinxAtStartPar
To avoid line wrapping, the report should be viewed on wide enough terminal; roughly 112 or chars or more.

\sphinxAtStartPar
For convenience after report is generated, the input files can be automatically moved to a save
direcory, left where they are or removed. A typical sequents of events is to save
the email reports, run dmarc\sphinxhyphen{}rpt.  By auto moving (or removing) the input files, makes it simpler
when doing the next batch of dmarc reports.

\sphinxAtStartPar
Then save all the raw .eml files into \textasciitilde{}/dmarc/reports and run before running the report

\begin{sphinxVerbatim}[commandchars=\\\{\}]
dmarc\PYGZhy{}rpt
\end{sphinxVerbatim}

\sphinxAtStartPar
All attachments from dmarc email reports would be saved into “\textasciitilde{}/dmarc/saved/2023\sphinxhyphen{}01”
in this example.


\subsection{tls\sphinxhyphen{}rpt Usage}
\label{\detokenize{README:tls-rpt-usage}}
\sphinxAtStartPar
tls\sphinxhyphen{}rpt works in a similar way to dmarc\sphinxhyphen{}rpt, except it operates on TLS\sphinxhyphen{}RPT (compresses) xml inputs.

\sphinxAtStartPar
Command line options are shown first in parens below, followed by
the corresponding config version in square brackets, if available.


\subsection{Common Options}
\label{\detokenize{README:common-options}}
\sphinxAtStartPar
These apply to both dmarc\sphinxhyphen{}rpt and tls\sphinxhyphen{}rpt
\begin{itemize}
\item {} 
\sphinxAtStartPar
(\sphinxstyleemphasis{\sphinxhyphen{}h, \textendash{}help})
Help for command line options.

\item {} 
\sphinxAtStartPar
(\sphinxstyleemphasis{\sphinxhyphen{}d, \textendash{}dir}) {[}\sphinxstyleemphasis{dir = /path/xxx/}{]}

\sphinxAtStartPar
Allows specifying the directory with the dmarc report files to be processed.
The directory holding the report files (.eml, .xml, .gz or .zip)
By default, dir is the current directory.

\item {} 
\sphinxAtStartPar
(\sphinxstyleemphasis{\sphinxhyphen{}k, \textendash{}keep})  {[}\sphinxstyleemphasis{keep = true}{]}

\sphinxAtStartPar
Prevent the \sphinxstyleemphasis{.eml} being removed after the attached xml reports are extracted.

\item {} 
\sphinxAtStartPar
(\sphinxstyleemphasis{\sphinxhyphen{}thm, \textendash{}theme})

\sphinxAtStartPar
Report is now in color.
Default theme is ‘dark’. Theme can be ‘light’ ‘dark’ or ‘none’, which turns off color report.

\item {} 
\sphinxAtStartPar
(\sphinxstyleemphasis{\sphinxhyphen{}v, \textendash{}verb})

\sphinxAtStartPar
More verbose output

\item {} 
\sphinxAtStartPar
(\sphinxstyleemphasis{\sphinxhyphen{}ifd, \textendash{}inp\_file\_disp})  {[}\sphinxstyleemphasis{inp\_file\_disp = save}{]}

\sphinxAtStartPar
Input file disposition options one of : none,save,delete
If set to save then all input files (xml, compressed xml and any kept eml files) are moved
to directory specified by \sphinxstyleemphasis{inp\_files\_save\_dir}.

\item {} 
\sphinxAtStartPar
(\sphinxstyleemphasis{\sphinxhyphen{}ifsd, \textendash{}inp\_files\_save\_dir})

\sphinxAtStartPar
When \sphinxstyleemphasis{inp\_file\_disp} is set, then input files are moved to this directory after report
is generated.  Files are saved by year\sphinxhyphen{}month under the save directory

\item {} 
\sphinxAtStartPar
(\sphinxstyleemphasis{ips, \textendash{}dom\_ips}) {[}\sphinxstyleemphasis{dom\_ips = {[}‘1.1.1.0/24’, ‘2.2.2.16/29’}{]}

\sphinxAtStartPar
Comma separated list of IPs / CIDRs for your own domains. When used in config file
format as array of IP strings.

\end{itemize}


\subsection{dmarc\sphinxhyphen{}rpt Specific Options}
\label{\detokenize{README:dmarc-rpt-specific-options}}
\sphinxAtStartPar
These are only applicable for dmarc\sphinxhyphen{}rpt.
\begin{itemize}
\item {} 
\sphinxAtStartPar
(\sphinxstyleemphasis{\sphinxhyphen{}ips, \textendash{}dom\_ips})  {[}\sphinxstyleemphasis{dom\_ips = {[}ip, cidr, … {]}}{]}

\sphinxAtStartPar
Set the ips for your own domain(s), which will then be colored to make them easy to spot.
Command line option is just comma separated list \sphinxhyphen{} no square brackets like config file.

\item {} 
\sphinxAtStartPar
(\sphinxstyleemphasis{fdm, \textendash{}dmarc\_fails})
\begin{quote}

\sphinxAtStartPar
Only include dmarc failures in report
\end{quote}

\item {} 
\sphinxAtStartPar
(\sphinxstyleemphasis{fdk, \textendash{}dkim\_fails})
\begin{quote}

\sphinxAtStartPar
Only include dkim failures in report
\end{quote}

\item {} 
\sphinxAtStartPar
(\sphinxstyleemphasis{fsp, \textendash{}spf\_fails})
\begin{quote}

\sphinxAtStartPar
Only include spf failures in report
\end{quote}

\end{itemize}


\section{Saving Email Reports From Email Client}
\label{\detokenize{README:saving-email-reports-from-email-client}}
\sphinxAtStartPar
In most mail clients, such as thunderbird,  one can select multiple email reports and
then use \sphinxstyleemphasis{File \sphinxhyphen{}\textgreater{} Save As} to save the email files into a directory of your choosing.
Each email gets saved with a \sphinxstyleemphasis{.eml} extension.


\chapter{Appendix}
\label{\detokenize{README:appendix}}

\section{Dependencies}
\label{\detokenize{README:dependencies}}\begin{itemize}
\item {} 
\sphinxAtStartPar
Run Time :
* python (3.13 or later)

\item {} 
\sphinxAtStartPar
Building Package:
* git
* wheel (aka python\sphinxhyphen{}wheel)
* build (aka python\sphinxhyphen{}build)
* installer (aka python\sphinxhyphen{}installer)
* poetry (aka python\sphinxhyphen{}poetry)
\sphinxhyphen{} rsync

\item {} 
\sphinxAtStartPar
Optional for building docs:
\begin{itemize}
\item {} 
\sphinxAtStartPar
sphinx

\item {} 
\sphinxAtStartPar
texlive\sphinxhyphen{}latexextra  (archlinux packaguing of texlive tools)

\end{itemize}

\end{itemize}


\section{Philosophy}
\label{\detokenize{README:philosophy}}
\sphinxAtStartPar
We follow the \sphinxstyleemphasis{live at head commit} philosophy. This means we recommend using the
latest commit on git master branch. We also provide git tags.

\sphinxAtStartPar
This approach is also taken by Google %
\begin{footnote}[1]\sphinxAtStartFootnote
\sphinxurl{https://github.com/google/googletest}
%
\end{footnote} %
\begin{footnote}[2]\sphinxAtStartFootnote
\sphinxurl{https://abseil.io/about/philosophy\#upgrade-support}
%
\end{footnote}.


\section{License}
\label{\detokenize{README:license}}
\sphinxAtStartPar
Created by Gene C. and licensed under the terms of the MIT license.
\begin{itemize}
\item {} 
\sphinxAtStartPar
SPDX\sphinxhyphen{}License\sphinxhyphen{}Identifier: MIT

\item {} 
\sphinxAtStartPar
Copyright (c) 2023, Gene C

\end{itemize}

\sphinxstepscope


\chapter{SMTP tls\sphinxhyphen{}rpt}
\label{\detokenize{Readme-TLS:smtp-tls-rpt}}\label{\detokenize{Readme-TLS::doc}}

\section{Overview}
\label{\detokenize{Readme-TLS:overview}}
\sphinxAtStartPar
Generate a human readable tls report from one or more standard tls report files.
These reports are used for a email domain with support for either DANE or MTA\sphinxhyphen{}STS or both.


\subsection{Usage}
\label{\detokenize{Readme-TLS:usage}}
\sphinxAtStartPar
Run from command line:
.. code\sphinxhyphen{}block:: bash
\begin{quote}

\sphinxAtStartPar
tls\sphinxhyphen{}rpt
\end{quote}

\sphinxAtStartPar
Generates reports from one or more emailed tls reports. Similar to
dmarc\sphinxhyphen{}rpt, the tool can consume email files (.eml) or the json attachments (plain or compressed)
delivered as part of the usual mts\sphinxhyphen{}sts reports \sphinxhyphen{} and in directory specified by \sphinxstyleemphasis{inp\_files\_save\_dir}.

\sphinxAtStartPar
\sphinxstyleemphasis{tls\sphinxhyphen{}rpt} is provided as part of the dmarc\_report package


\subsubsection{Background}
\label{\detokenize{Readme-TLS:background}}
\sphinxAtStartPar
TLS Reports are oprionally generated for a mail domaain when so requested by a TXT record in
the domain’s DNS. The tool parses and summarizes such email reports for human consumption.

\sphinxAtStartPar
SMTP TLS reporting is described by {[}RFC 8460{]} %
\begin{footnote}[1]\sphinxAtStartFootnote
TLS Report {[}RFRC 8460{]} \sphinxurl{https://www.rfc-editor.org/rfc/rfc8460.txt}
%
\end{footnote} where it summarizes:

\begin{sphinxVerbatim}[commandchars=\\\{\}]
\PYG{n}{A} \PYG{n}{number} \PYG{n}{of} \PYG{n}{protocols} \PYG{n}{exist} \PYG{k}{for} \PYG{n}{establishing} \PYG{n}{encrypted} \PYG{n}{channels}
\PYG{n}{between} \PYG{n}{SMTP} \PYG{n}{Mail} \PYG{n}{Transfer} \PYG{n}{Agents} \PYG{p}{(}\PYG{n}{MTAs}\PYG{p}{)}\PYG{p}{,} \PYG{n}{including} \PYG{n}{STARTTLS}\PYG{p}{,} \PYG{n}{DNS}\PYG{o}{\PYGZhy{}}
\PYG{n}{Based} \PYG{n}{Authentication} \PYG{n}{of} \PYG{n}{Named} \PYG{n}{Entities} \PYG{p}{(}\PYG{n}{DANE}\PYG{p}{)} \PYG{n}{TLSA}\PYG{p}{,} \PYG{o+ow}{and} \PYG{n}{MTA} \PYG{n}{Strict}
\PYG{n}{Transport} \PYG{n}{Security} \PYG{p}{(}\PYG{n}{MTA}\PYG{o}{\PYGZhy{}}\PYG{n}{STS}\PYG{p}{)}\PYG{o}{.}
\end{sphinxVerbatim}

\sphinxAtStartPar
MTA\sphinxhyphen{}STS, is explained by {[}RFC 8641{]} %
\begin{footnote}[2]\sphinxAtStartFootnote
MTA\sphinxhyphen{}STS {[}RFC 8461{]} \sphinxurl{https://www.rfc-editor.org/rfc/rfc8461.txt}
%
\end{footnote} where it is summarized:

\begin{sphinxVerbatim}[commandchars=\\\{\}]
\PYG{n}{SMTP} \PYG{n}{MTA} \PYG{n}{Strict} \PYG{n}{Transport} \PYG{n}{Security} \PYG{p}{(}\PYG{n}{MTA}\PYG{o}{\PYGZhy{}}\PYG{n}{STS}\PYG{p}{)} \PYG{o+ow}{is} \PYG{n}{a} \PYG{n}{mechanism} \PYG{n}{enabling}
\PYG{n}{mail} \PYG{n}{service} \PYG{n}{providers} \PYG{p}{(}\PYG{n}{SPs}\PYG{p}{)} \PYG{n}{to} \PYG{n}{declare} \PYG{n}{their} \PYG{n}{ability} \PYG{n}{to} \PYG{n}{receive}
\PYG{n}{Transport} \PYG{n}{Layer} \PYG{n}{Security} \PYG{p}{(}\PYG{n}{TLS}\PYG{p}{)} \PYG{n}{secure} \PYG{n}{SMTP} \PYG{n}{connections} \PYG{o+ow}{and} \PYG{n}{to} \PYG{n}{specify}
\PYG{n}{whether} \PYG{n}{sending} \PYG{n}{SMTP} \PYG{n}{servers} \PYG{n}{should} \PYG{n}{refuse} \PYG{n}{to} \PYG{n}{deliver} \PYG{n}{to} \PYG{n}{MX} \PYG{n}{hosts}
\PYG{n}{that} \PYG{n}{do} \PYG{o+ow}{not} \PYG{n}{offer} \PYG{n}{TLS} \PYG{k}{with} \PYG{n}{a} \PYG{n}{trusted} \PYG{n}{server} \PYG{n}{certificate}\PYG{o}{.}
\end{sphinxVerbatim}

\sphinxAtStartPar
while DANE is documented in {[}RFC 6698{]} %
\begin{footnote}[3]\sphinxAtStartFootnote
DANE {[}RFC 6698{]} \sphinxurl{https://www.rfc-editor.org/rfc/rfc6698.txt}
%
\end{footnote}, {[}RFC 7671{]} %
\begin{footnote}[4]\sphinxAtStartFootnote
DANE {[}RFC 7671{]} \sphinxurl{https://www.rfc-editor.org/rfc/rfc7671.txt}
%
\end{footnote} and {[}RFC 7672{]} %
\begin{footnote}[5]\sphinxAtStartFootnote
DANE SMTP {[}RFC 7672{]} \sphinxurl{https://www.rfc-editor.org/rfc/rfc7672.txt}
%
\end{footnote}

\begin{sphinxVerbatim}[commandchars=\\\{\}]
\PYG{n}{Encrypted} \PYG{n}{communication} \PYG{n}{on} \PYG{n}{the} \PYG{n}{Internet} \PYG{n}{often} \PYG{n}{uses} \PYG{n}{Transport} \PYG{n}{Layer}
\PYG{n}{Security} \PYG{p}{(}\PYG{n}{TLS}\PYG{p}{)}\PYG{p}{,} \PYG{n}{which} \PYG{n}{depends} \PYG{n}{on} \PYG{n}{third} \PYG{n}{parties} \PYG{n}{to} \PYG{n}{certify} \PYG{n}{the} \PYG{n}{keys}
\PYG{n}{used}\PYG{o}{.}  \PYG{n}{This} \PYG{n}{document} \PYG{n}{improves} \PYG{n}{on} \PYG{n}{that} \PYG{n}{situation} \PYG{n}{by} \PYG{n}{enabling} \PYG{n}{the}
\PYG{n}{administrators} \PYG{n}{of} \PYG{n}{domain} \PYG{n}{names} \PYG{n}{to} \PYG{n}{specify} \PYG{n}{the} \PYG{n}{keys} \PYG{n}{used} \PYG{o+ow}{in} \PYG{n}{that}
\PYG{n}{domain}\PYG{l+s+s1}{\PYGZsq{}}\PYG{l+s+s1}{s TLS servers.  This requires matching improvements in TLS}
\PYG{n}{client} \PYG{n}{software}\PYG{p}{,} \PYG{n}{but} \PYG{n}{no} \PYG{n}{change} \PYG{o+ow}{in} \PYG{n}{TLS} \PYG{n}{server} \PYG{n}{software}
\end{sphinxVerbatim}


\subsubsection{Discussion}
\label{\detokenize{Readme-TLS:discussion}}
\sphinxAtStartPar
To receive TLS reports requires a DNS record requesting a TLS report along with
either a DANE TLSA record or MTA\sphinxhyphen{}STS. MTA\sphinxhyphen{}STS requires both a policy and
and a DNS record.


\subsection{TLS Report DNS Record}
\label{\detokenize{Readme-TLS:tls-report-dns-record}}\begin{quote}

\sphinxAtStartPar
Example

\sphinxAtStartPar
\_smtp.\_tls.example.org IN TXT “v=TLSRPTv1;rua=mailto:\sphinxhref{mailto:tlsrpt@example.com}{tlsrpt@example.com}”

\sphinxAtStartPar
The TLS reports will be sent to the email provided by the string following \sphinxstyleemphasis{rua=}.
In this example reports would be sent to \sphinxstyleemphasis{tlsrpt@example.com}.
\end{quote}


\subsection{MTA\sphinxhyphen{}STS}
\label{\detokenize{Readme-TLS:mta-sts}}
\sphinxAtStartPar
Requieres both a DNS record and a policy file available from the email’s domain web server.

\sphinxAtStartPar
Policy file example to be provided by web server:

\begin{sphinxVerbatim}[commandchars=\\\{\}]
\PYG{n}{https}\PYG{p}{:}\PYG{o}{/}\PYG{o}{/}\PYG{n}{mta}\PYG{o}{\PYGZhy{}}\PYG{n}{sts}\PYG{o}{.}\PYG{n}{example}\PYG{o}{.}\PYG{n}{com}\PYG{o}{/}\PYG{o}{.}\PYG{n}{well}\PYG{o}{\PYGZhy{}}\PYG{n}{known}\PYG{o}{/}\PYG{n}{mta}\PYG{o}{\PYGZhy{}}\PYG{n}{sts}\PYG{o}{.}\PYG{n}{txt}
\end{sphinxVerbatim}

\sphinxAtStartPar
The policy mode can be set to \sphinxstyleemphasis{enforce} or \sphinxstyleemphasis{testing}.
Example \sphinxstyleemphasis{mta\sphinxhyphen{}sts.txt} file:

\begin{sphinxVerbatim}[commandchars=\\\{\}]
\PYG{n}{version}\PYG{p}{:} \PYG{n}{STSv1}
\PYG{n}{mode}\PYG{p}{:} \PYG{n}{enforce}
\PYG{n}{mx}\PYG{p}{:} \PYG{n}{example}\PYG{o}{.}\PYG{n}{com}
\PYG{n}{mx}\PYG{p}{:} \PYGZbs{}\PYG{o}{*}\PYG{o}{.}\PYG{n}{example}\PYG{o}{.}\PYG{n}{com}
\PYG{n}{max\PYGZus{}age}\PYG{p}{:} \PYG{l+m+mi}{1296000}
\end{sphinxVerbatim}

\sphinxAtStartPar
DNS TXT record example:

\begin{sphinxVerbatim}[commandchars=\\\{\}]
\PYGZus{}mta\PYGZhy{}sts.example.org.  IN TXT “v=STSv1; id=202301011200;”
\end{sphinxVerbatim}


\subsection{DANE TLSA}
\label{\detokenize{Readme-TLS:dane-tlsa}}
\sphinxAtStartPar
DNS record example:

\begin{sphinxVerbatim}[commandchars=\\\{\}]
\PYG{n}{\PYGZus{}25}\PYG{o}{.}\PYG{n}{\PYGZus{}tcp}\PYG{o}{.}\PYG{n}{example}\PYG{o}{.}\PYG{n}{com}\PYG{o}{.} \PYG{n}{TLSA} \PYG{l+m+mi}{3} \PYG{l+m+mi}{1} \PYG{l+m+mi}{1} \PYG{p}{(}\PYG{n}{xxx}\PYG{p}{)}
\end{sphinxVerbatim}

\sphinxAtStartPar
where xxx would be the appropriate public key hash.


\subsubsection{Using tls\sphinxhyphen{}rpt}
\label{\detokenize{Readme-TLS:using-tls-rpt}}
\sphinxAtStartPar
Save all tls email reports into a directory.
Change to the directory containing one or more dmarc report files and simply run
.. code\sphinxhyphen{}block:: back
\begin{quote}

\sphinxAtStartPar
tls\sphinxhyphen{}rpt
\end{quote}

\sphinxAtStartPar
Using the \textendash{}dir option (or setging the config option \sphinxstyleemphasis{dir}) makes unnecessary to change directories before
running the report.

\sphinxAtStartPar
Any email files, those ending with \sphinxstyleemphasis{.eml} will be processed first. These are assumed to
contain the dmarc report as a mime attachment. The attachment is extracted from such email
files.

\sphinxAtStartPar
Subsequently, all remaining files are read and processed. The tool processes all json
and gzip/zip compressed json tls report files and produces a human readable report.

\sphinxAtStartPar
Any file with extension \sphinxstyleemphasis{.eml} is treated as an email file.

\sphinxAtStartPar
For convenience after report is generated, the input files can be automatically moved to a save
direcory, left where they are or removed. A typical sequents of eveents is to save
the email reports, run dmarc\sphinxhyphen{}rpt.  By auto moving (or removing) the input files, makes it simpler
when doing the next batch of dmarc reports.

\sphinxAtStartPar
For example, you might save all .eml files in same directory and with config settings:

\begin{sphinxVerbatim}[commandchars=\\\{\}]
\PYG{n+nb}{dir} \PYG{o}{=} \PYG{l+s+s2}{\PYGZdq{}}\PYG{l+s+s2}{\PYGZti{}/tlsrpt/reports}\PYG{l+s+s2}{\PYGZdq{}}
\PYG{n}{inp\PYGZus{}files\PYGZus{}disp} \PYG{o}{=} \PYG{l+s+s2}{\PYGZdq{}}\PYG{l+s+s2}{save}\PYG{l+s+s2}{\PYGZdq{}}
\PYG{n}{inp\PYGZus{}files\PYGZus{}save\PYGZus{}dir} \PYG{o}{=} \PYG{l+s+s2}{\PYGZdq{}}\PYG{l+s+s2}{../saved}\PYG{l+s+s2}{\PYGZdq{}}
\end{sphinxVerbatim}

\sphinxAtStartPar
Then save all the raw .eml files into \textasciitilde{}/tlsrpt/reports and run

\begin{sphinxVerbatim}[commandchars=\\\{\}]
\PYG{n}{tls}\PYG{o}{\PYGZhy{}}\PYG{n}{rpt}
\end{sphinxVerbatim}

\sphinxAtStartPar
All attachments from email reports would be saved into “\textasciitilde{}/tlsrpt/saved/2023\sphinxhyphen{}01”
in this example.


\subsection{tls\sphinxhyphen{}rpt Options}
\label{\detokenize{Readme-TLS:tls-rpt-options}}
\sphinxAtStartPar
Options are read first from config files then command line. Config files are read
from \sphinxstyleemphasis{/etc/dmarc\_report/config\sphinxhyphen{}tls} then \sphinxstyleemphasis{\textasciitilde{}/.config/dmarc\_report/config\sphinxhyphen{}tls}.  Config files
are in standard TOML format.

\sphinxAtStartPar
Config settings use corresponding command line option:

\begin{sphinxVerbatim}[commandchars=\\\{\}]
\PYG{n}{long}\PYG{o}{\PYGZhy{}}\PYG{n}{option} \PYG{o}{=} \PYG{n}{xxx}\PYG{o}{.}
\end{sphinxVerbatim}

\sphinxAtStartPar
e.g. to set data report dir in config use

\begin{sphinxVerbatim}[commandchars=\\\{\}]
\PYG{n+nb}{dir} \PYG{o}{=} \PYG{o}{/}\PYG{n}{foo}\PYG{o}{/}\PYG{n}{goo}\PYG{o}{/}\PYG{n}{other}
\end{sphinxVerbatim}

\sphinxAtStartPar
The command line options are shown first in parens followed by
corresponding config in square brackets if available.
\begin{itemize}
\item {} 
\sphinxAtStartPar
(\sphinxstyleemphasis{\sphinxhyphen{}d, \textendash{}dir}) {[}\sphinxstyleemphasis{dir = /some/path}{]}

\sphinxAtStartPar
Allows specifying the directory with the dmarc report files to be processed.
The directory holding the report files (.eml, .json, .gz or .zip)
By default, dir is the current directory.

\item {} 
\sphinxAtStartPar
(\sphinxstyleemphasis{\sphinxhyphen{}k, \textendash{}keep}) {[}\sphinxstyleemphasis{keep = true}{]}

\sphinxAtStartPar
Prevent the \sphinxstyleemphasis{.eml} being removed after the attached xml reports are extracted.

\item {} 
\sphinxAtStartPar
(\sphinxstyleemphasis{\sphinxhyphen{}thm, \textendash{}theme} )

\sphinxAtStartPar
Report is now in color.
Default theme is ‘dark’. Theme can be ‘light’ ‘dark’ or ‘none’, which turns off color report.

\item {} 
\sphinxAtStartPar
(\sphinxstyleemphasis{\sphinxhyphen{}ifd, \textendash{}inp\_file\_disp})

\sphinxAtStartPar
Input file disposition options one of : none,save,delete
If set to save then all input files (xml, compressed xml and any kept eml files) are moved
to directory specified by \sphinxstyleemphasis{inp\_files\_save\_dir}.

\item {} 
\sphinxAtStartPar
(\sphinxstyleemphasis{\sphinxhyphen{}ifsd, \textendash{}inp\_files\_save\_dir})

\sphinxAtStartPar
When \sphinxstyleemphasis{inp\_file\_disp} is set, then input files are moved to this directory after report
is generated.  Files are saved by year\sphinxhyphen{}month under the save directory

\item {} 
\sphinxAtStartPar
(\sphinxstyleemphasis{\sphinxhyphen{}h, \textendash{}help})

\sphinxAtStartPar
Help for command line options.

\end{itemize}


\subsection{Saving Email Reports From Email Client}
\label{\detokenize{Readme-TLS:saving-email-reports-from-email-client}}
\sphinxAtStartPar
In most mail clients, such as thunderbird,  one can select multiple email reports and
then use \sphinxstyleemphasis{File \sphinxhyphen{}\textgreater{} Save As} to save the email files into a directory of your choosing.
Each email gets saved with a \sphinxstyleemphasis{.eml} extension.


\subsection{License}
\label{\detokenize{Readme-TLS:license}}
\sphinxAtStartPar
Created by Gene C. It is licensed under the terms of the MIT license.
\begin{itemize}
\item {} 
\sphinxAtStartPar
SPDX\sphinxhyphen{}License\sphinxhyphen{}Identifier: MIT

\item {} 
\sphinxAtStartPar
Copyright (c) 2023, Gene C

\end{itemize}

\sphinxstepscope


\chapter{Changelog}
\label{\detokenize{Changelog:changelog}}\label{\detokenize{Changelog::doc}}
\sphinxAtStartPar
\sphinxstylestrong{{[}4.13.1{]} —\textendash{} 2025\sphinxhyphen{}02\sphinxhyphen{}23}

\begin{sphinxVerbatim}[commandchars=\\\{\}]
\PYG{n}{Change} \PYG{n}{to} \PYG{n}{py}\PYG{o}{\PYGZhy{}}\PYG{n}{cidr} \PYG{n}{package} \PYG{k}{for} \PYG{n}{network} \PYG{n}{tools}\PYG{o}{.}
\PYG{n}{Update} \PYG{n}{README}
\PYG{n}{update} \PYG{n}{Changelog}\PYG{o}{.}\PYG{n}{rst}
\end{sphinxVerbatim}

\sphinxAtStartPar
\sphinxstylestrong{{[}4.12.5{]} —\textendash{} 2025\sphinxhyphen{}01\sphinxhyphen{}11}

\begin{sphinxVerbatim}[commandchars=\\\{\}]
\PYG{n}{Ensure} \PYG{n}{python} \PYG{n}{version} \PYG{n}{requirement} \PYG{o+ow}{is} \PYG{n}{consistent} \PYG{p}{(}\PYG{n}{README}\PYG{p}{,} \PYG{n}{pyproject}\PYG{p}{,} \PYG{n}{PKGBUILD}\PYG{p}{,} \PYG{n}{requirements}\PYG{p}{)}
\PYG{n}{update} \PYG{n}{Changelog}\PYG{o}{.}\PYG{n}{rst}
\end{sphinxVerbatim}

\sphinxAtStartPar
\sphinxstylestrong{{[}4.12.4{]} —\textendash{} 2024\sphinxhyphen{}12\sphinxhyphen{}31}

\begin{sphinxVerbatim}[commandchars=\\\{\}]
\PYG{n}{Add} \PYG{n}{git} \PYG{n}{signing} \PYG{n}{key} \PYG{n}{to} \PYG{n}{Arch} \PYG{n}{Package}
\PYG{n}{update} \PYG{n}{Changelog}\PYG{o}{.}\PYG{n}{rst}
\end{sphinxVerbatim}

\sphinxAtStartPar
\sphinxstylestrong{{[}4.12.3{]} —\textendash{} 2024\sphinxhyphen{}12\sphinxhyphen{}31}

\begin{sphinxVerbatim}[commandchars=\\\{\}]
\PYG{n}{typo}
\PYG{n}{update} \PYG{n}{Changelog}\PYG{o}{.}\PYG{n}{rst}
\end{sphinxVerbatim}

\sphinxAtStartPar
\sphinxstylestrong{{[}4.12.2{]} —\textendash{} 2024\sphinxhyphen{}12\sphinxhyphen{}31}

\begin{sphinxVerbatim}[commandchars=\\\{\}]
\PYG{n}{Add} \PYG{n}{validpgpkeys} \PYG{n}{to} \PYG{n}{PKGBUILD}
\PYG{n}{update} \PYG{n}{Changelog}\PYG{o}{.}\PYG{n}{rst}
\end{sphinxVerbatim}

\sphinxAtStartPar
\sphinxstylestrong{{[}4.12.1{]} —\textendash{} 2024\sphinxhyphen{}12\sphinxhyphen{}31}

\begin{sphinxVerbatim}[commandchars=\\\{\}]
\PYG{n}{All} \PYG{n}{git} \PYG{n}{tags} \PYG{n}{are} \PYG{n}{now} \PYG{n}{signed}\PYG{o}{.}
\PYG{n}{Update} \PYG{n}{SPDX} \PYG{n}{tags}
\PYG{n}{update} \PYG{n}{Changelog}\PYG{o}{.}\PYG{n}{rst}
\end{sphinxVerbatim}

\sphinxAtStartPar
\sphinxstylestrong{{[}4.12.0{]} —\textendash{} 2024\sphinxhyphen{}11\sphinxhyphen{}28}

\begin{sphinxVerbatim}[commandchars=\\\{\}]
\PYG{n}{Handle} \PYG{n}{another} \PYG{n}{seconds} \PYG{n+nb}{format} \PYG{o+ow}{in} \PYG{n}{xml} \PYG{n}{file}
\PYG{n}{update} \PYG{n}{Changelog}\PYG{o}{.}\PYG{n}{rst}
\end{sphinxVerbatim}

\sphinxAtStartPar
\sphinxstylestrong{{[}4.11.0{]} —\textendash{} 2024\sphinxhyphen{}10\sphinxhyphen{}22}

\begin{sphinxVerbatim}[commandchars=\\\{\}]
\PYG{n}{Additional} \PYG{n+nb}{input} \PYG{n}{protections} \PYG{o+ow}{in} \PYG{n}{cidr} \PYG{n}{utils}
\PYG{n}{update} \PYG{n}{Changelog}\PYG{o}{.}\PYG{n}{rst}
\end{sphinxVerbatim}

\sphinxAtStartPar
\sphinxstylestrong{{[}4.10.0{]} —\textendash{} 2024\sphinxhyphen{}10\sphinxhyphen{}22}

\begin{sphinxVerbatim}[commandchars=\\\{\}]
\PYG{n}{Bug} \PYG{n}{fix} \PYG{n}{when} \PYG{n}{no} \PYG{l+s+s2}{\PYGZdq{}}\PYG{l+s+s2}{dom\PYGZus{}ips}\PYG{l+s+s2}{\PYGZdq{}} \PYG{n+nb}{set}\PYG{o}{.} \PYG{n}{Resolves} \PYG{n}{issue} \PYG{c+c1}{\PYGZsh{}2 reported by @g4242}
\PYG{n}{update} \PYG{n}{Changelog}\PYG{o}{.}\PYG{n}{rst}
\end{sphinxVerbatim}

\sphinxAtStartPar
\sphinxstylestrong{{[}4.9.0{]} —\textendash{} 2024\sphinxhyphen{}10\sphinxhyphen{}20}

\begin{sphinxVerbatim}[commandchars=\\\{\}]
\PYG{n}{remove} \PYG{n}{dead} \PYG{n}{code}
\PYG{n}{update} \PYG{n}{Changelog}\PYG{o}{.}\PYG{n}{rst}
\end{sphinxVerbatim}

\sphinxAtStartPar
\sphinxstylestrong{{[}4.8.0{]} —\textendash{} 2024\sphinxhyphen{}10\sphinxhyphen{}20}

\begin{sphinxVerbatim}[commandchars=\\\{\}]
\PYG{n}{For} \PYG{n}{completeness}\PYG{p}{,} \PYG{n}{Handle} \PYG{n}{ip} \PYG{n}{address} \PYG{n}{of} \PYG{n}{form} \PYG{n}{ip}\PYG{o}{/}\PYG{n}{prefix}
\PYG{n}{update} \PYG{n}{Changelog}\PYG{o}{.}\PYG{n}{rst}
\end{sphinxVerbatim}

\sphinxAtStartPar
\sphinxstylestrong{{[}4.7.0{]} —\textendash{} 2024\sphinxhyphen{}10\sphinxhyphen{}19}

\begin{sphinxVerbatim}[commandchars=\\\{\}]
\PYG{n}{Now} \PYG{n}{use} \PYG{n}{python} \PYG{l+m+mi}{3}\PYG{n}{s} \PYG{n}{ipaddress} \PYG{n}{module} \PYG{n}{instead} \PYG{n}{of} \PYG{n}{netaddr}\PYG{o}{.}
  \PYG{n}{Its} \PYG{n}{faster} \PYG{o+ow}{and} \PYG{n}{we} \PYG{n}{no} \PYG{n}{longer} \PYG{n}{require} \PYG{l+m+mi}{3}\PYG{n}{rd} \PYG{n}{party} \PYG{n}{module}
\PYG{n}{Require} \PYG{n}{python} \PYG{n}{version} \PYG{l+m+mf}{3.11} \PYG{o+ow}{or} \PYG{n}{later}
\PYG{n}{update} \PYG{n}{Changelog}\PYG{o}{.}\PYG{n}{rst}
\end{sphinxVerbatim}

\sphinxAtStartPar
\sphinxstylestrong{{[}4.6.0{]} —\textendash{} 2024\sphinxhyphen{}08\sphinxhyphen{}29}

\begin{sphinxVerbatim}[commandchars=\\\{\}]
\PYG{n}{Switch} \PYG{n}{to} \PYG{n}{lxml} \PYG{k}{for} \PYG{n}{better} \PYG{n}{handling} \PYG{n}{of} \PYG{n}{namespaces} \PYG{n}{found} \PYG{o+ow}{in} \PYG{n}{some} \PYG{n}{reports}
\PYG{n}{Now} \PYG{n}{handle} \PYG{n}{namespaces} \PYG{p}{(}\PYG{n}{e}\PYG{o}{.}\PYG{n}{g}\PYG{o}{.} \PYG{n}{GMX} \PYG{n}{uses} \PYG{n}{them}\PYG{p}{)}
\PYG{n}{update} \PYG{n}{Changelog}\PYG{o}{.}\PYG{n}{rst}
\end{sphinxVerbatim}

\sphinxAtStartPar
\sphinxstylestrong{{[}4.3.1{]} —\textendash{} 2023\sphinxhyphen{}12\sphinxhyphen{}26}

\begin{sphinxVerbatim}[commandchars=\\\{\}]
\PYG{n}{Add} \PYG{n}{missing} \PYG{n}{dateutil} \PYG{n}{to} \PYG{n}{depends} \PYG{o+ow}{in} \PYG{n}{PKGBUILD}
\PYG{n}{update} \PYG{n}{Changelog}\PYG{o}{.}\PYG{n}{rst}
\end{sphinxVerbatim}

\sphinxAtStartPar
\sphinxstylestrong{{[}4.3.0{]} —\textendash{} 2023\sphinxhyphen{}12\sphinxhyphen{}10}

\begin{sphinxVerbatim}[commandchars=\\\{\}]
\PYG{n}{Add} \PYG{n}{support} \PYG{k}{for} \PYG{n}{extracting} \PYG{n}{reports} \PYG{k+kn}{from}\PYG{+w}{ }\PYG{n+nn}{multiple} \PYG{n}{emails} \PYG{n}{saved} \PYG{n}{into} \PYG{n}{an} \PYG{n}{mbox} \PYG{n}{file} \PYG{o}{\PYGZhy{}} \PYG{n}{evolution} \PYG{n}{saves} \PYG{n}{emails} \PYG{n}{this} \PYG{n}{way}
\PYG{n}{update} \PYG{n}{Changelog}\PYG{o}{.}\PYG{n}{rst}
\end{sphinxVerbatim}

\sphinxAtStartPar
\sphinxstylestrong{{[}4.2.0{]} —\textendash{} 2023\sphinxhyphen{}11\sphinxhyphen{}28}

\begin{sphinxVerbatim}[commandchars=\\\{\}]
\PYG{n}{Handle} \PYG{n}{badly} \PYG{n}{formed} \PYG{n}{dmarc} \PYG{n}{report} \PYG{k}{with} \PYG{n}{missing} \PYG{n}{date} \PYG{n+nb}{range}
\PYG{n}{Switch} \PYG{n}{python} \PYG{n}{build} \PYG{n}{backend} \PYG{n}{to} \PYG{n}{hatch} \PYG{p}{(}\PYG{n}{was} \PYG{n}{poetry}\PYG{p}{)}
\PYG{n}{update} \PYG{n}{CHANGELOG}\PYG{o}{.}\PYG{n}{md}
\end{sphinxVerbatim}

\sphinxAtStartPar
\sphinxstylestrong{{[}4.0.0{]} —\textendash{} 2023\sphinxhyphen{}10\sphinxhyphen{}29}

\begin{sphinxVerbatim}[commandchars=\\\{\}]
\PYG{n}{Improve} \PYG{n}{tls}\PYG{o}{\PYGZhy{}}\PYG{n}{rpt}
      \PYG{n}{Show} \PYG{n}{policy} \PYG{n}{name} \PYG{p}{(}\PYG{n}{tlsa}\PYG{p}{,} \PYG{n}{sts}\PYG{p}{,} \PYG{n}{none}\PYG{p}{)}
      \PYG{n}{Show} \PYG{n}{count} \PYG{n}{of} \PYG{n}{each} \PYG{n}{failure} \PYG{n}{result} \PYG{n+nb}{type}
      \PYG{n}{Now} \PYG{n}{checks} \PYG{n+nb}{all} \PYG{l+s+s2}{\PYGZdq{}}\PYG{l+s+s2}{policies}\PYG{l+s+s2}{\PYGZdq{}} \PYG{n}{returned} \PYG{o+ow}{in} \PYG{n}{the} \PYG{n}{json} \PYG{n}{report}\PYG{o}{.}
      \PYG{n}{Add} \PYG{n}{date} \PYG{n}{ranges} \PYG{n}{to} \PYG{n}{report}
\PYG{n}{update} \PYG{n}{CHANGELOG}\PYG{o}{.}\PYG{n}{md}
\end{sphinxVerbatim}

\sphinxAtStartPar
\sphinxstylestrong{{[}3.10.0{]} —\textendash{} 2023\sphinxhyphen{}09\sphinxhyphen{}27}

\begin{sphinxVerbatim}[commandchars=\\\{\}]
\PYG{n}{Reorganize} \PYG{n}{documentation} \PYG{n}{under} \PYG{n}{Docs} \PYG{o+ow}{and} \PYG{n}{migrate} \PYG{n}{to} \PYG{n}{restructured} \PYG{n}{text}
\PYG{n}{Nicer} \PYG{n}{formatting} \PYG{o+ow}{in} \PYG{n}{README}\PYG{o}{\PYGZhy{}}\PYG{n}{tls}\PYG{o}{.}\PYG{n}{rst}
\PYG{n}{update} \PYG{n}{CHANGELOG}\PYG{o}{.}\PYG{n}{md}
\end{sphinxVerbatim}

\sphinxAtStartPar
\sphinxstylestrong{{[}3.9.2{]} —\textendash{} 2023\sphinxhyphen{}07\sphinxhyphen{}14}

\begin{sphinxVerbatim}[commandchars=\\\{\}]
\PYG{n}{Change} \PYG{n}{to} \PYG{l+m+mf}{3.9}\PYG{l+m+mf}{.2}
\PYG{n}{update} \PYG{n}{CHANGELOG}\PYG{o}{.}\PYG{n}{md}
\end{sphinxVerbatim}

\sphinxAtStartPar
\sphinxstylestrong{{[}3.9.1{]} —\textendash{} 2023\sphinxhyphen{}07\sphinxhyphen{}14}

\begin{sphinxVerbatim}[commandchars=\\\{\}]
\PYG{n}{With} \PYG{n}{updated} \PYG{n}{README}\PYG{o}{\PYGZhy{}}\PYG{n}{tls}\PYG{o}{.}\PYG{n}{rst} \PYG{n}{this} \PYG{n}{time}
\PYG{n}{update} \PYG{n}{CHANGELOG}\PYG{o}{.}\PYG{n}{md}
\end{sphinxVerbatim}

\sphinxAtStartPar
\sphinxstylestrong{{[}3.9.0{]} —\textendash{} 2023\sphinxhyphen{}07\sphinxhyphen{}14}

\begin{sphinxVerbatim}[commandchars=\\\{\}]
\PYG{n}{Update} \PYG{n}{README} \PYG{k}{with} \PYG{n}{better} \PYG{n}{description} \PYG{n}{of} \PYG{n}{TLS} \PYG{n}{Report} \PYG{o+ow}{and} \PYG{n}{use} \PYG{n}{rst}
\PYG{n}{update} \PYG{n}{CHANGELOG}\PYG{o}{.}\PYG{n}{md}
\end{sphinxVerbatim}

\sphinxAtStartPar
\sphinxstylestrong{{[}3.8.0{]} —\textendash{} 2023\sphinxhyphen{}07\sphinxhyphen{}09}

\begin{sphinxVerbatim}[commandchars=\\\{\}]
\PYG{n}{Add} \PYG{n+nb}{any} \PYG{n}{failure} \PYG{n}{details} \PYG{n}{to} \PYG{n}{tls} \PYG{n}{report}
\PYG{n}{update} \PYG{n}{CHANGELOG}\PYG{o}{.}\PYG{n}{md}
\end{sphinxVerbatim}

\sphinxAtStartPar
\sphinxstylestrong{{[}3.7.1{]} —\textendash{} 2023\sphinxhyphen{}05\sphinxhyphen{}18}

\begin{sphinxVerbatim}[commandchars=\\\{\}]
\PYG{n}{Update} \PYG{n}{build} \PYG{n}{info} \PYG{o+ow}{in} \PYG{n}{README}
\PYG{n}{update} \PYG{n}{CHANGELOG}\PYG{o}{.}\PYG{n}{md}
\end{sphinxVerbatim}

\sphinxAtStartPar
\sphinxstylestrong{{[}3.7.0{]} —\textendash{} 2023\sphinxhyphen{}05\sphinxhyphen{}18}

\begin{sphinxVerbatim}[commandchars=\\\{\}]
\PYG{n}{install}\PYG{p}{:} \PYG{n}{switch} \PYG{k+kn}{from}\PYG{+w}{ }\PYG{n+nn}{pip} \PYG{n}{to} \PYG{n}{python} \PYG{n}{installer} \PYG{n}{package}\PYG{o}{.} \PYG{n}{This} \PYG{n}{adds} \PYG{n}{optimized} \PYG{n}{bytecode}
\PYG{n}{update} \PYG{n}{CHANGELOG}\PYG{o}{.}\PYG{n}{md}
\end{sphinxVerbatim}

\sphinxAtStartPar
\sphinxstylestrong{{[}3.6.3{]} —\textendash{} 2023\sphinxhyphen{}05\sphinxhyphen{}18}

\begin{sphinxVerbatim}[commandchars=\\\{\}]
\PYG{n}{PKGBUILD}\PYG{p}{:} \PYG{n}{add} \PYG{n}{python}\PYG{o}{\PYGZhy{}}\PYG{n}{build} \PYG{n}{to} \PYG{n}{makedepends}
\PYG{n}{update} \PYG{n}{CHANGELOG}\PYG{o}{.}\PYG{n}{md}
\end{sphinxVerbatim}

\sphinxAtStartPar
\sphinxstylestrong{{[}3.6.2{]} —\textendash{} 2023\sphinxhyphen{}05\sphinxhyphen{}18}

\begin{sphinxVerbatim}[commandchars=\\\{\}]
\PYG{n}{PKGBUILD}\PYG{p}{:} \PYG{n}{build} \PYG{n}{wheel} \PYG{n}{back} \PYG{n}{to} \PYG{n}{using} \PYG{n}{python} \PYG{o}{\PYGZhy{}}\PYG{n}{m} \PYG{n}{build} \PYG{n}{instead} \PYG{n}{of} \PYG{n}{poetry}
\PYG{n}{update} \PYG{n}{CHANGELOG}\PYG{o}{.}\PYG{n}{md}
\end{sphinxVerbatim}

\sphinxAtStartPar
\sphinxstylestrong{{[}3.6.1{]} —\textendash{} 2023\sphinxhyphen{}05\sphinxhyphen{}17}

\begin{sphinxVerbatim}[commandchars=\\\{\}]
\PYG{n}{Simplify} \PYG{n}{Arch} \PYG{n}{PKGBUILD} \PYG{o+ow}{and} \PYG{n}{more} \PYG{n}{closely} \PYG{n}{follow} \PYG{n}{arch} \PYG{n}{guidelines}
\PYG{n}{update} \PYG{n}{CHANGELOG}\PYG{o}{.}\PYG{n}{md}
\end{sphinxVerbatim}

\sphinxAtStartPar
\sphinxstylestrong{{[}3.6.0{]} —\textendash{} 2023\sphinxhyphen{}04\sphinxhyphen{}29}

\begin{sphinxVerbatim}[commandchars=\\\{\}]
\PYG{n}{Handle} \PYG{n}{exceptions} \PYG{k+kn}{from}\PYG{+w}{ }\PYG{n+nn}{bad} \PYG{n}{XML} \PYG{n}{report} \PYG{n}{files}
\PYG{n}{update} \PYG{n}{CHANGELOG}\PYG{o}{.}\PYG{n}{md}
\end{sphinxVerbatim}

\sphinxAtStartPar
\sphinxstylestrong{{[}3.5.0{]} —\textendash{} 2023\sphinxhyphen{}01\sphinxhyphen{}21}

\begin{sphinxVerbatim}[commandchars=\\\{\}]
Remove duplicate line in options class \PYGZhy{} has no effect
update CHANGELOG.md
\end{sphinxVerbatim}

\sphinxAtStartPar
\sphinxstylestrong{{[}3.4.0{]} —\textendash{} 2023\sphinxhyphen{}01\sphinxhyphen{}17}

\begin{sphinxVerbatim}[commandchars=\\\{\}]
Turn off debug \PYGZhy{} accidently left on with last release! So sorry
typo in README\PYGZhy{}mta\PYGZhy{}sts.md
update CHANGELOG.md
\end{sphinxVerbatim}

\sphinxAtStartPar
\sphinxstylestrong{{[}3.3.0{]} —\textendash{} 2023\sphinxhyphen{}01\sphinxhyphen{}09}

\begin{sphinxVerbatim}[commandchars=\\\{\}]
\PYG{n}{More} \PYG{n}{info} \PYG{n}{about} \PYG{n}{selectors} \PYG{n}{including} \PYG{n}{missing} \PYG{p}{(}\PYG{l+s+s2}{\PYGZdq{}}\PYG{l+s+s2}{\PYGZhy{}}\PYG{l+s+s2}{\PYGZdq{}}\PYG{p}{)}
\PYG{n}{update} \PYG{n}{CHANGELOG}\PYG{o}{.}\PYG{n}{md}
\end{sphinxVerbatim}

\sphinxAtStartPar
\sphinxstylestrong{{[}3.2.0{]} —\textendash{} 2023\sphinxhyphen{}01\sphinxhyphen{}09}

\begin{sphinxVerbatim}[commandchars=\\\{\}]
\PYG{n}{Add} \PYG{n}{more} \PYG{n}{info} \PYG{n}{about} \PYG{n}{dkim} \PYG{n}{selectors} \PYG{n}{typically} \PYG{k+kn}{from}\PYG{+w}{ }\PYG{n+nn}{forwarded} \PYG{n}{mail}
\PYG{n}{update} \PYG{n}{CHANGELOG}\PYG{o}{.}\PYG{n}{md}
\end{sphinxVerbatim}

\sphinxAtStartPar
\sphinxstylestrong{{[}3.1.0{]} —\textendash{} 2023\sphinxhyphen{}01\sphinxhyphen{}09}

\begin{sphinxVerbatim}[commandchars=\\\{\}]
\PYG{n}{Sort} \PYG{n}{short} \PYG{n}{dkim} \PYG{n}{selector} \PYG{n}{tags} \PYG{n}{before} \PYG{n}{printing}
\PYG{n}{tweak} \PYG{n}{readme} \PYG{k}{for} \PYG{n}{new} \PYG{n}{tls}\PYG{o}{\PYGZhy{}}\PYG{n}{rpt} \PYG{n}{tool}
\PYG{n}{update} \PYG{n}{CHANGELOG}\PYG{o}{.}\PYG{n}{md}
\end{sphinxVerbatim}

\sphinxAtStartPar
\sphinxstylestrong{{[}3.0.0{]} —\textendash{} 2023\sphinxhyphen{}01\sphinxhyphen{}07}

\begin{sphinxVerbatim}[commandchars=\\\{\}]
\PYG{n}{Refactor} \PYG{n}{code} \PYG{n}{some}\PYG{o}{.}
\PYG{n}{Add} \PYG{n}{new} \PYG{n}{tls}\PYG{o}{\PYGZhy{}}\PYG{n}{rpt} \PYG{n}{to} \PYG{n}{generate} \PYG{n}{reports} \PYG{k}{for} \PYG{n}{MTA}\PYG{o}{\PYGZhy{}}\PYG{n}{STS} \PYG{n}{TLS} \PYG{n}{reports}
\PYG{n}{update} \PYG{n}{CHANGELOG}\PYG{o}{.}\PYG{n}{md}
\end{sphinxVerbatim}

\sphinxAtStartPar
\sphinxstylestrong{{[}2.3.0{]} —\textendash{} 2023\sphinxhyphen{}01\sphinxhyphen{}07}

\begin{sphinxVerbatim}[commandchars=\\\{\}]
\PYG{n}{Bug} \PYG{n}{fix} \PYG{o}{\PYGZhy{}} \PYG{n}{clean} \PYG{n}{up} \PYG{n}{went} \PYG{n}{too} \PYG{n}{far} \PYG{n}{added} \PYG{n}{silly} \PYG{n+nb}{print} \PYG{n}{bug} \PYG{o}{\PYGZhy{}} \PYG{n}{so} \PYG{n}{sorry}
\PYG{n}{tidy} \PYG{n}{README}\PYG{p}{,} \PYG{n}{add} \PYG{n}{SPDX} \PYG{n}{license} \PYG{n}{line} \PYG{n}{to} \PYG{n}{missed} \PYG{n}{file}
\PYG{n}{update} \PYG{n}{CHANGELOG}\PYG{o}{.}\PYG{n}{md}
\end{sphinxVerbatim}

\sphinxAtStartPar
\sphinxstylestrong{{[}2.2.1{]} —\textendash{} 2023\sphinxhyphen{}01\sphinxhyphen{}06}

\begin{sphinxVerbatim}[commandchars=\\\{\}]
\PYG{n}{Use} \PYG{n}{SPDX} \PYG{n}{licensing}\PYG{o}{.}
\PYG{n}{Lint} \PYG{o+ow}{and} \PYG{n}{tidy}
\PYG{n}{Fix} \PYG{n}{description} \PYG{n}{of} \PYG{n+nb}{input} \PYG{n}{file} \PYG{n}{disposition} \PYG{n}{to} \PYG{n}{show} \PYG{n}{none}\PYG{p}{,}\PYG{n}{save}\PYG{p}{,}\PYG{n}{delete}
\PYG{n}{update} \PYG{n}{CHANGELOG}\PYG{o}{.}\PYG{n}{md}
\end{sphinxVerbatim}

\sphinxAtStartPar
\sphinxstylestrong{{[}2.2.0{]} —\textendash{} 2023\sphinxhyphen{}01\sphinxhyphen{}05}

\begin{sphinxVerbatim}[commandchars=\\\{\}]
\PYG{n}{Add} \PYG{n}{option} \PYG{k}{for} \PYG{n}{disposition} \PYG{n}{of} \PYG{n+nb}{input} \PYG{n}{files} \PYG{n}{after} \PYG{n}{report} \PYG{o+ow}{is} \PYG{n}{generated}\PYG{o}{.}
   \PYG{o}{\PYGZhy{}}\PYG{o}{\PYGZhy{}}\PYG{n}{inp\PYGZus{}files\PYGZus{}disp} \PYG{n}{can} \PYG{n}{be} \PYG{n}{none}\PYG{p}{,} \PYG{n}{save} \PYG{o+ow}{or} \PYG{n}{delete}\PYG{o}{.}  \PYG{n}{Default} \PYG{o+ow}{is} \PYG{n}{none}\PYG{o}{.}
   \PYG{o}{\PYGZhy{}}\PYG{o}{\PYGZhy{}}\PYG{n}{inp\PYGZus{}files\PYGZus{}save\PYGZus{}dir} \PYG{n}{specifies} \PYG{n}{where} \PYG{n}{to} \PYG{n}{save} \PYG{n+nb}{input} \PYG{n}{files} \PYG{n}{when} \PYG{n}{disposition} \PYG{o+ow}{is} \PYG{l+s+s2}{\PYGZdq{}}\PYG{l+s+s2}{save}\PYG{l+s+s2}{\PYGZdq{}}
\PYG{n}{update} \PYG{n}{CHANGELOG}\PYG{o}{.}\PYG{n}{md}
\end{sphinxVerbatim}

\sphinxAtStartPar
\sphinxstylestrong{{[}2.1.0{]} —\textendash{} 2023\sphinxhyphen{}01\sphinxhyphen{}03}

\begin{sphinxVerbatim}[commandchars=\\\{\}]
\PYG{n}{Right} \PYG{n}{align} \PYG{n}{numbers}
\PYG{n}{small} \PYG{n}{tweak} \PYG{n}{to} \PYG{n}{README}
\PYG{n}{update} \PYG{n}{CHANGELOG}\PYG{o}{.}\PYG{n}{md}
\end{sphinxVerbatim}

\sphinxAtStartPar
\sphinxstylestrong{{[}2.0.0{]} —\textendash{} 2023\sphinxhyphen{}01\sphinxhyphen{}03}

\begin{sphinxVerbatim}[commandchars=\\\{\}]
\PYG{n}{Fix} \PYG{n}{bug} \PYG{n}{where} \PYG{n}{grand} \PYG{n}{total} \PYG{n}{missed} \PYG{n}{orgs} \PYG{k}{with} \PYG{l+m+mi}{1} \PYG{n}{IP}
\PYG{n}{Add} \PYG{n}{color} \PYG{n}{report}\PYG{p}{,} \PYG{n}{default} \PYG{n}{theme} \PYG{o+ow}{is} \PYG{n}{dark}\PYG{o}{.} \PYG{n}{Can} \PYG{n}{be} \PYG{n}{light}\PYG{p}{,} \PYG{n}{dark} \PYG{o+ow}{or} \PYG{n}{none} \PYG{n}{to} \PYG{n}{turn} \PYG{n}{color} \PYG{n}{off}
\PYG{n}{Add} \PYG{n}{support} \PYG{k}{for} \PYG{n}{config} \PYG{n}{files}\PYG{p}{:} \PYG{o}{/}\PYG{n}{etc}\PYG{o}{/}\PYG{n}{dmarc\PYGZus{}report}\PYG{o}{/}\PYG{n}{config} \PYG{o}{\PYGZhy{}} \PYG{o}{\PYGZti{}}\PYG{o}{.}\PYG{n}{config}\PYG{o}{/}\PYG{n}{dmarc\PYGZus{}report}\PYG{o}{/}\PYG{n}{config}
  \PYG{n}{Config} \PYG{n}{file} \PYG{o+ow}{is} \PYG{n}{TOML} \PYG{n+nb}{format} \PYG{n}{where} \PYG{n}{each} \PYG{n}{variable} \PYG{o+ow}{is} \PYG{n}{the} \PYG{n}{long\PYGZus{}option} \PYG{n}{name}\PYG{p}{:}
  \PYG{n}{e}\PYG{o}{.}\PYG{n}{g}\PYG{o}{.} \PYG{n+nb}{dir} \PYG{o}{=} \PYG{l+s+s2}{\PYGZdq{}}\PYG{l+s+s2}{/a/b/dmarc\PYGZus{}stuff}\PYG{l+s+s2}{\PYGZdq{}}
\PYG{n}{Add} \PYG{n}{new} \PYG{n}{option} \PYG{n}{to} \PYG{n+nb}{set} \PYG{n}{your} \PYG{n}{IP} \PYG{o+ow}{or} \PYG{n}{CIDR} \PYG{n}{blocks} \PYG{o}{\PYGZhy{}} \PYG{n}{this} \PYG{n}{will} \PYG{n}{allow} \PYG{n}{your} \PYG{n}{own} \PYG{n}{IPs} \PYG{n}{to} \PYG{n}{be} \PYG{n}{colored}
  \PYG{n}{Makes} \PYG{n}{it} \PYG{n}{easy} \PYG{n}{to} \PYG{n}{spot} \PYG{n}{mail} \PYG{n}{generated} \PYG{k+kn}{from}\PYG{+w}{ }\PYG{n+nn}{your} \PYG{n}{own} \PYG{n}{IP} \PYG{n}{vs} \PYG{n}{mail} \PYG{n}{lists} \PYG{n}{etc}
\PYG{n}{update} \PYG{n}{CHANGELOG}\PYG{o}{.}\PYG{n}{md}
\end{sphinxVerbatim}

\sphinxAtStartPar
\sphinxstylestrong{{[}1.3.1{]} —\textendash{} 2023\sphinxhyphen{}01\sphinxhyphen{}03}

\begin{sphinxVerbatim}[commandchars=\\\{\}]
\PYG{n}{Improve} \PYG{n}{report} \PYG{n+nb}{format} \PYG{n}{a} \PYG{n}{bit}
\PYG{n}{typo}
\PYG{n}{small} \PYG{n}{README} \PYG{n}{tweak}
\PYG{n}{update} \PYG{n}{CHANGELOG}\PYG{o}{.}\PYG{n}{md}
\end{sphinxVerbatim}

\sphinxAtStartPar
\sphinxstylestrong{{[}1.3.0{]} —\textendash{} 2023\sphinxhyphen{}01\sphinxhyphen{}02}

\begin{sphinxVerbatim}[commandchars=\\\{\}]
\PYG{n}{silly} \PYG{n}{bug} \PYG{k}{with} \PYG{n}{multipart} \PYG{n}{accidenlty} \PYG{n}{ignoring} \PYG{n}{report} \PYG{n}{file}
\PYG{n}{update} \PYG{n}{CHANGELOG}\PYG{o}{.}\PYG{n}{md}
\end{sphinxVerbatim}

\sphinxAtStartPar
\sphinxstylestrong{{[}1.2.1{]} —\textendash{} 2023\sphinxhyphen{}01\sphinxhyphen{}02}

\begin{sphinxVerbatim}[commandchars=\\\{\}]
\PYG{n}{remove} \PYG{n}{reference} \PYG{n}{to} \PYG{n}{ripmime} \PYG{o}{\PYGZhy{}} \PYG{n}{no} \PYG{n}{longer} \PYG{n}{needed} \PYG{n}{now} \PYG{n}{that} \PYG{n}{we} \PYG{n}{handle} \PYG{n}{mime} \PYG{n}{attachments} \PYG{n}{ourselves}
\PYG{n}{update} \PYG{n}{CHANGELOG}\PYG{o}{.}\PYG{n}{md}
\end{sphinxVerbatim}

\sphinxAtStartPar
\sphinxstylestrong{{[}1.2.0{]} —\textendash{} 2023\sphinxhyphen{}01\sphinxhyphen{}02}

\begin{sphinxVerbatim}[commandchars=\\\{\}]
\PYG{n}{Fix} \PYG{n}{bug} \PYG{k}{with} \PYG{n}{some} \PYG{n}{multipart} \PYG{n}{mime} \PYG{n}{email} \PYG{k+kn}{from}\PYG{+w}{ }\PYG{n+nn}{some} \PYG{n}{reporters}
\PYG{n}{update} \PYG{n}{CHANGELOG}\PYG{o}{.}\PYG{n}{md}
\end{sphinxVerbatim}

\sphinxAtStartPar
\sphinxstylestrong{{[}1.1.0{]} —\textendash{} 2023\sphinxhyphen{}01\sphinxhyphen{}02}

\begin{sphinxVerbatim}[commandchars=\\\{\}]
\PYG{o}{*}\PYG{o}{.}\PYG{n}{eml}\PYG{o}{*} \PYG{n}{files} \PYG{n}{are} \PYG{n}{now} \PYG{n}{removed} \PYG{n}{after} \PYG{n}{the} \PYG{n}{dmarc} \PYG{n}{report} \PYG{o+ow}{is} \PYG{n}{extracted}\PYG{o}{.}
   \PYG{n}{Use} \PYG{n}{option} \PYG{o}{*}\PYG{o}{\PYGZhy{}}\PYG{n}{k}\PYG{p}{,} \PYG{o}{\PYGZhy{}}\PYG{o}{\PYGZhy{}}\PYG{n}{keep}\PYG{o}{*} \PYG{n}{to} \PYG{n}{prevent} \PYG{n}{the} \PYG{o}{*}\PYG{o}{.}\PYG{n}{eml}\PYG{o}{*} \PYG{n}{being} \PYG{n}{removed}
\PYG{n}{update} \PYG{n}{CHANGELOG}\PYG{o}{.}\PYG{n}{md}
\end{sphinxVerbatim}

\sphinxAtStartPar
\sphinxstylestrong{{[}1.0.0{]} —\textendash{} 2023\sphinxhyphen{}01\sphinxhyphen{}02}

\begin{sphinxVerbatim}[commandchars=\\\{\}]
\PYG{n}{Added} \PYG{n}{support} \PYG{n}{to} \PYG{n}{extract} \PYG{n}{dmarc} \PYG{n}{reports} \PYG{k+kn}{from}\PYG{+w}{ }\PYG{n+nn}{mime} \PYG{n}{attachments} \PYG{o+ow}{in} \PYG{n}{email} \PYG{n}{files}
    \PYG{n}{Added} \PYG{n}{option} \PYG{o}{*}\PYG{o}{\PYGZhy{}}\PYG{n}{d}\PYG{p}{,} \PYG{o}{\PYGZhy{}}\PYG{o}{\PYGZhy{}}\PYG{n+nb}{dir}\PYG{o}{*} \PYG{n}{to} \PYG{n}{specify} \PYG{n}{the} \PYG{n}{directory} \PYG{n}{containing} \PYG{n}{report} \PYG{n}{files}
\PYG{n}{more} \PYG{n}{readme} \PYG{n}{tweaks}
\PYG{n}{tweak} \PYG{n}{readme}
\PYG{n}{update} \PYG{n}{CHANGELOG}\PYG{o}{.}\PYG{n}{md}
\end{sphinxVerbatim}

\sphinxAtStartPar
\sphinxstylestrong{{[}0.9.1{]} —\textendash{} 2023\sphinxhyphen{}01\sphinxhyphen{}02}

\begin{sphinxVerbatim}[commandchars=\\\{\}]
\PYG{n}{Add} \PYG{n}{note} \PYG{n}{on} \PYG{n}{handling} \PYG{n}{email} \PYG{n}{reports} \PYG{n}{efficiently} \PYG{n}{to} \PYG{n}{README}
\PYG{n}{remove} \PYG{n}{unused} \PYG{n}{file}
\PYG{n}{update} \PYG{n}{CHANGELOG}\PYG{o}{.}\PYG{n}{md}
\end{sphinxVerbatim}

\sphinxAtStartPar
\sphinxstylestrong{{[}0.9.0{]} —\textendash{} 2023\sphinxhyphen{}01\sphinxhyphen{}01}

\begin{sphinxVerbatim}[commandchars=\\\{\}]
\PYG{n}{Small} \PYG{n}{tweak} \PYG{n}{to} \PYG{n}{report} \PYG{n}{output}
\PYG{n}{fix} \PYG{n}{typo}
\PYG{n}{update} \PYG{n}{CHANGELOG}\PYG{o}{.}\PYG{n}{md}
\end{sphinxVerbatim}

\sphinxAtStartPar
\sphinxstylestrong{{[}0.8.1{]} —\textendash{} 2023\sphinxhyphen{}01\sphinxhyphen{}01}

\begin{sphinxVerbatim}[commandchars=\\\{\}]
\PYG{n}{update} \PYG{n}{readme}
\PYG{n}{update} \PYG{n}{CHANGELOG}\PYG{o}{.}\PYG{n}{md}
\end{sphinxVerbatim}

\sphinxAtStartPar
\sphinxstylestrong{{[}0.8.0{]} —\textendash{} 2023\sphinxhyphen{}01\sphinxhyphen{}01}

\begin{sphinxVerbatim}[commandchars=\\\{\}]
\PYG{n}{bump} \PYG{n}{vers} \PYG{n}{to} \PYG{l+m+mf}{0.8}\PYG{l+m+mf}{.0}
\PYG{n}{update} \PYG{n}{CHANGELOG}\PYG{o}{.}\PYG{n}{md}
\end{sphinxVerbatim}

\sphinxAtStartPar
\sphinxstylestrong{{[}0.7.0{]} —\textendash{} 2023\sphinxhyphen{}01\sphinxhyphen{}01}

\begin{sphinxVerbatim}[commandchars=\\\{\}]
\PYG{n}{prep} \PYG{k}{for} \PYG{n}{release}
\end{sphinxVerbatim}

\sphinxAtStartPar
\sphinxstylestrong{{[}0.6.0{]} —\textendash{} 2023\sphinxhyphen{}01\sphinxhyphen{}01}

\begin{sphinxVerbatim}[commandchars=\\\{\}]
\PYG{n}{initial} \PYG{n}{commit}
\end{sphinxVerbatim}

\sphinxstepscope


\chapter{MIT License}
\label{\detokenize{License:mit-license}}\label{\detokenize{License::doc}}
\sphinxAtStartPar
Copyright © 2023 Gene C

\sphinxAtStartPar
Permission is hereby granted, free of charge, to any person obtaining a copy
of this software and associated documentation files (the “Software”), to deal
in the Software without restriction, including without limitation the rights
to use, copy, modify, merge, publish, distribute, sublicense, and/or sell
copies of the Software, and to permit persons to whom the Software is
furnished to do so, subject to the following conditions:

\sphinxAtStartPar
The above copyright notice and this permission notice shall be included in all
copies or substantial portions of the Software.

\sphinxAtStartPar
THE SOFTWARE IS PROVIDED “AS IS”, WITHOUT WARRANTY OF ANY KIND, EXPRESS OR
IMPLIED, INCLUDING BUT NOT LIMITED TO THE WARRANTIES OF MERCHANTABILITY,
FITNESS FOR A PARTICULAR PURPOSE AND NONINFRINGEMENT. IN NO EVENT SHALL THE
AUTHORS OR COPYRIGHT HOLDERS BE LIABLE FOR ANY CLAIM, DAMAGES OR OTHER
LIABILITY, WHETHER IN AN ACTION OF CONTRACT, TORT OR OTHERWISE, ARISING FROM,
OUT OF OR IN CONNECTION WITH THE SOFTWARE OR THE USE OR OTHER DEALINGS IN THE
SOFTWARE.

\sphinxstepscope


\chapter{How to help with this project}
\label{\detokenize{Contributing:how-to-help-with-this-project}}\label{\detokenize{Contributing::doc}}
\sphinxAtStartPar
Thank you for your interest in improving this project.
This project is open\sphinxhyphen{}source under the MIT license.


\section{Important resources}
\label{\detokenize{Contributing:important-resources}}\begin{itemize}
\item {} 
\sphinxAtStartPar
\sphinxhref{https://github.com/gene-git/dmarc\_report}{Git Repo}

\end{itemize}


\section{Reporting Bugs or feature requests}
\label{\detokenize{Contributing:reporting-bugs-or-feature-requests}}
\sphinxAtStartPar
Please report bugs on the issue tracker in the git repo.
To make the report as useful as possible, please include
\begin{itemize}
\item {} 
\sphinxAtStartPar
operating system used

\item {} 
\sphinxAtStartPar
version of python

\item {} 
\sphinxAtStartPar
explanation of the problem or enhancement request.

\end{itemize}


\section{Code Changes}
\label{\detokenize{Contributing:code-changes}}
\sphinxAtStartPar
If you make code changes, please update the documentation if
it’s appropriate.
\begin{quote}
\end{quote}

\sphinxstepscope


\chapter{Contributor Covenant Code of Conduct}
\label{\detokenize{Code-of-conduct:contributor-covenant-code-of-conduct}}\label{\detokenize{Code-of-conduct:code-of-conduct}}\label{\detokenize{Code-of-conduct::doc}}

\section{Our Pledge}
\label{\detokenize{Code-of-conduct:our-pledge}}
\sphinxAtStartPar
In the interest of fostering an open and welcoming environment, we as
contributors and maintainers pledge to making participation in our project and
our community a harassment\sphinxhyphen{}free experience for everyone, regardless of age, body
size, disability, ethnicity, sex characteristics, gender identity and
expression, level of experience, education, socio\sphinxhyphen{}economic status, nationality,
personal appearance, race, religion, or sexual identity and orientation.


\section{Our Standards}
\label{\detokenize{Code-of-conduct:our-standards}}
\sphinxAtStartPar
Examples of behavior that contributes to creating a positive environment
include:
\begin{itemize}
\item {} 
\sphinxAtStartPar
Using welcoming and inclusive language

\item {} 
\sphinxAtStartPar
Being respectful of differing viewpoints and experiences

\item {} 
\sphinxAtStartPar
Gracefully accepting constructive criticism

\item {} 
\sphinxAtStartPar
Focusing on what is best for the community

\item {} 
\sphinxAtStartPar
Showing empathy towards other community members

\end{itemize}

\sphinxAtStartPar
Examples of unacceptable behavior by participants include:
\begin{itemize}
\item {} 
\sphinxAtStartPar
The use of sexualized language or imagery and unwelcome sexual attention or
advances

\item {} 
\sphinxAtStartPar
Trolling, insulting/derogatory comments, and personal or political attacks

\item {} 
\sphinxAtStartPar
Public or private harassment

\item {} 
\sphinxAtStartPar
Publishing others’ private information, such as a physical or electronic
address, without explicit permission

\item {} 
\sphinxAtStartPar
Other conduct which could reasonably be considered inappropriate in a
professional setting

\end{itemize}


\section{Our Responsibilities}
\label{\detokenize{Code-of-conduct:our-responsibilities}}
\sphinxAtStartPar
Maintainers are responsible for clarifying the standards of acceptable behavior
and are expected to take appropriate and fair corrective action in response to
any instances of unacceptable behavior.

\sphinxAtStartPar
Maintainers have the right and responsibility to remove, edit, or reject
comments, commits, code, wiki edits, issues, and other contributions that are
not aligned to this Code of Conduct, or to ban temporarily or permanently any
contributor for other behaviors that they deem inappropriate, threatening,
offensive, or harmful.


\section{Scope}
\label{\detokenize{Code-of-conduct:scope}}
\sphinxAtStartPar
This Code of Conduct applies both within project spaces and in public spaces
when an individual is representing the project or its community. Examples of
representing a project or community include using an official project e\sphinxhyphen{}mail
address, posting via an official social media account, or acting as an appointed
representative at an online or offline event. Representation of a project may be
further defined and clarified by project maintainers.


\section{Enforcement}
\label{\detokenize{Code-of-conduct:enforcement}}
\sphinxAtStartPar
Instances of abusive, harassing, or otherwise unacceptable behavior may be
reported by contacting the project team at \textless{}\sphinxhref{mailto:arch@sapience.com}{arch@sapience.com}\textgreater{}.
All complaints will be reviewed and investigated
and will result in a response that is deemed necessary and appropriate
to the circumstances. The Code of Conduct Committee is obligated to
maintain confidentiality with regard to the reporter of an incident.
Further details of specific enforcement policies may be posted
separately.


\section{Attribution}
\label{\detokenize{Code-of-conduct:attribution}}
\sphinxAtStartPar
This Code of Conduct is adapted from the Contributor Covenant, version 1.4,
available at \sphinxurl{https://www.contributor-covenant.org/version/1/4/code-of-conduct.html}


\section{Interpretation}
\label{\detokenize{Code-of-conduct:interpretation}}
\sphinxAtStartPar
The interpretation of this document is at the discretion of the project team.


\chapter{Indices and tables}
\label{\detokenize{index:indices-and-tables}}\begin{itemize}
\item {} 
\sphinxAtStartPar
\DUrole{xref}{\DUrole{std}{\DUrole{std-ref}{genindex}}}

\item {} 
\sphinxAtStartPar
\DUrole{xref}{\DUrole{std}{\DUrole{std-ref}{modindex}}}

\item {} 
\sphinxAtStartPar
\DUrole{xref}{\DUrole{std}{\DUrole{std-ref}{search}}}

\end{itemize}



\renewcommand{\indexname}{Index}
\printindex
\end{document}